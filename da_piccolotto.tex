% Technische Universität Dresden
% Fakultät Informatik
% Institut für Software- und Multimediatechnik
% Seniorprofessur für Multimediatechnik
%
% Example document demonstrating the usage of mmthesis.sty
% 2012-10-26 Andreas Rümpel
% 
% ### build hints (% = filename of tex file) ###
% pdflatex: pdflatex %.tex
% biber: biber % (biber is a modern backend for bibtex, http://biblatex-biber.sourceforge.net)
% glossaries and acronyms: makeglossaries %
%
% Das LATEX2e-Sündenregister: ftp://ftp.dante.de/tex-archive/info/l2tabu/german/l2tabu.pdf
% KOMAScript-Guide: ftp://ftp.dante.de/tex-archive/macros/latex/contrib/koma-script/scrguide.påådf
% Einige typographische Grundregeln und ihre Umsetzung in LaTeX: http://www2.informatik.hu-berlin.de/sv/lehre/typographie.pdf

\documentclass[
	headsepline,
	footsepline,
	fontsize=12pt,
	%draft, % use this for finding overfull boxes
	%parskip, % use this for alternative paragraph formatting
	bibliography=totoc
]{scrbook}

\usepackage{mmthesis}
\addbibresource{library.bib} % put name of bib file here with extension

%### switches
%\printoutput % make link colors black, leave deactivated for screen output

%### define metadata
\mmtype{Diplomarbeit} %Diplomarbeit|Großer Beleg|Bachelorarbeit|Masterarbeit|Seminararbeit
\mmtitle{Semantik-gestütztes Hilfesystem für ein komposites Informationsvisualisierungssystem}
\mmtshorttitle{Hilfesystem für komposite InfoVis}
\mmtauthor{Nikolaus Piccolotto}
\mmtsubmissionmonth{November 2013}
\mmtsubmissiondate{30. November 2013}
\mmtsupervisor{Dipl.-Medieninf. Martin Voigt}
%\mmtsupervisorii{Dipl.-Medieninf. Foo Bar} % Co-supervisor, optional

\mmthypersetup % has to be called after setting metadata

%### acronyms
\newacronym{PDF}{PDF}{Portable Document Format}
\newacronym{RCP}{RCP}{Rich Client Platform}
\newacronym{RIA}{RIA}{Rich Internet Application}
\newacronym{RELAXNG}{RELAX NG}{Regular Language Description for XML New Generation}
\newacronym{SGML}{SGML}{Standard Generalized Markup Language}
\newacronym{SWT}{SWT}{Standard Widget Toolkit}
\newacronym{WDC}{W3C}{World Wide Web Consortium}
\newacronym{WPF}{WPF}{Windows Presentation Foundation}
\newacronym{XPath}{XPath}{XML Path Language}
\newacronym{XHTML}{XHTML}{Extensible Hypertext Markup Language}
\newacronym{XML}{XML}{Extensible Markup Language}
% my acronyms
\newacronym{InfoVis}{InfoVis}{Informationsvisualisierung}

\begin{document}
\frontmatter
\pagenumbering{Roman}
\mmtfrontmatter

\listoffigures
\listoftables
\printglossary[type=\acronymtype,style=long,title=Abkürzungsverzeichnis,toctitle=Abkürzungsverzeichnis]
%\printglossary %Glossar

\mainmatter

% ###################################################
\chapter{Einleitung}
\label{chapter:einleitung}

% Hier kommt eine kurze Einführung von semantischen Daten, nehme ich an

% ###################################################
\section{Motivation}
\label{section:motivation}

% Hier wird generell eingeleitet, also vermutlich die Problematik zwischen semantischen Datensätzen und Endusern

% ###################################################
\section{Problemstellung und Zielsetzung}
\label{section:problemstellung}

% Hier wird quasi der Vizboard Workflow umrissen
% und erklärt, wo das eigentliche Problem liegt, nämlich im Information Overload beim letzten Schritt

\begin{figure}[htbp]
	\centering
	\includegraphics[width=0.75\textwidth]{images/vizboard_workflow.png}
	\caption{VizBoard Workflow}
	\label{figure:vizboard_workflow}
\end{figure}

% ###################################################
\section{Aufbau der Arbeit}
\label{section:aufbau}

% Aufbau der Arbeit erklären, kommt zum Schluss

% ###################################################
\chapter{Stand der Forschung und Technik}
\label{chapter:standderforschung}

Der folgende Abschnitt besteht aus drei Teilen. Zuerst wird die Aufgabenstellung in einem Szenario verdeutlicht (Abschnitt~\ref{section:szenario}). Daraus werden Anforderungen an das Hilfesystem abgeleitet (Abschnitt~\ref{section:anforderungsanalyse}) und danach die Grundlagen von semantischen Daten, Informationsvisualisierungen und Software Support erläutert (Abschnitt~\ref{section:grundlagen}).

% ###################################################
\section{Szenario}
\label{section:szenario}

% kurze einleitung noch mal

Wie in Kapitel~\ref{chapter:einleitung} erläutert, ist das komposite InfoVis-System Teil der webbasierten Anwendung VizBoard. Sie leitet den Benutzer in mehreren Schritten von der Auswahl eines Datensatzes zur finalen, kompositen Informationsvisualisierung. Im vorletzten Schritt wählt dieser mit Hilfe eines Facettenbrowsers geeignete Visualisierungskomponenten aus, welche danach angezeigt werden. Um die Problemstellung noch einmal zu verdeutlichen, wird im folgenden ein mögliches Szenario beschrieben.

% einführung der problemstellung
Anna möchte für ihr Biologiestudium mehr über die geografische Verteilung verschiedener Genvarationen herausfinden. Dazu sucht sie im Internet nach einem Datensatz, welchen sie auch findet. 
% unbekanntes format, kann es nirgends ordentlich öffen und selbst wenn es in excel ginge, wüsste sie nicht, welche charts sie am besten erstellen sollte
Leider ist er in einem für Anna unbekannten Format abgespeichert, nämlich OWL. Sie versucht die Datei mit Microsoft Excel und SPSS zu öffnen, weil sie keine anderen Programme zur Datenverarbeitung kennt, aber scheitert. Anna stellt fest, dass nur ihr Texteditor OWL öffnen und vernünftig darstellen kann. Als sie die Datei überfliegt, kann sie den Inhalt zwar erahnen, aber es ist einfach zu viel Text um ihn vollständig zu lesen. Davon abgesehen sind geografische Breite und Länge als Zahlenkombination keine anschauliche Repräsentation von Orten, auch Verteilungen von Werten sind so schwer ersichtlich. Anna würde viel Zeit aufwenden müssen um sehr wenig des Inhalts zu verstehen. Aber selbst wenn sie die Datei in Excel hätte öffnen können, hätte sie nicht gewusst, mit welchen Diagrammen die vorhandenen Daten am Besten verstanden würden. Anna hört von einem Freund, dass VizBoard gut geeignet ist, um semantische Datensätze anzusehen und probiert es aus.

% Der Benutzer ist laut unserem Rollenmodell weder Developer noch Visualisierungsexperte, d.h. er hat erstmal Schwierigkeiten zu erfassen, was hier überhaupt abgeht

Anna hat ihren Datensatz auch bei VizBoard gefunden und ist neugierig: Sie wählt eine Karte, ein Balkendiagramm, eine Tabelle und eine Treemap aus (Abbildung~\ref{figure:szenario-skizze}); kurz darauf werden ihr die Visualisierungskomponenten angezeigt. Anna benutzt VizBoard zum ersten Mal und macht außer Facebook und YouTube auch sonst nicht viel im Internet, das heißt sie ist zunächst von den vier unterschiedlichen Fenstern etwas überfordert.

\begin{figure}[htbp]
	\centering
	\includegraphics[width=0.75\textwidth]{images/szenario-skizze.png}
	\caption{Skizze der VizBoard Visualisierungskomponenten}
	\label{figure:szenario-skizze}
\end{figure}

VizBoard bietet Anna aber sofort eine einführende Übersicht und erklärt kurz die Darstellungsform und den Inhalt jeder Komponente. Eine denkbare Erklärung der Treemap wäre zum Beispiel:

\begin{quote}
Eine Treemap ist eine hierarchische Visualisierung, um Größenverhältnisse anschaulich zu machen. In dieser werden die Anzahl von Genvariationen pro geografischer Region dargestellt.
\end{quote}

% Was sind das für Fenster? Welche Komponente ist welche? Welche macht was? Wie kann ich sie bedienen? HUUUPS da ändert sich ja was obwohl ich dort nix gemacht hab! Wie hängen die zusammen? Was sind das für Daten, die dargestellt werden? 

Damit bekommt Anna einen Überblick über die verfügbaren Visualisierungen und weiß, welches Fenster welche Visualisierung enthält und für was diese gut sind. Nun möchte sie die Tabelle, in der die durch die Treemap visualisierten Daten stehen, nach der Spalte \enquote{Anzahl} sortieren. Anna sieht aber nicht, wie sie das machen soll, da in der Tabelle kein offensichtliches Kontrollelement wie z.\,B. ein Button vorhanden ist. Sie bemerkt ein Fragezeichen in der Titelleiste des Fensters und klickt darauf. Der verfügbare Viewport wird abgedunkelt und es erscheint ein neues Fenster, welches die verfügbaren Aktionen mit Hilfe von Text, Bildern und Animationen erklärt. Anna lernt, dass sie mit einem einfachen Linksklick auf den jeweiligen Kopf einer Tabellenspalte nach dieser sortieren kann und außerdem eine oder mehrere Zeilen auswählen kann. Sie sortiert die Tabelle wie gewollt und wählt die ersten drei Zeilen aus. Plötzlich verkleinert die Karte ihr Zoomlevel und Anna ist verwirrt: Sie hat nur mit der Tabelle interagiert und es bestand keine sichtbare Verbindung zwischen den beiden Fenstern. Allerdings wurde nach der Zeilenauswahl ein Pfeil von der Tabelle zur Karte gezeichnet, welcher mit einem Icon in Form eines Briefes versehen ist. Anna vermutet, dass doch irgendeine Verbindung zwischen den beiden Visualisierungen besteht und klickt auf den Brief. Ähnlich wie vorhin bei der Hilfe zur Tabelle wird der Viewport abgedunkelt und ein neues Fenster wird eingeblendet. Es erklärt die Kommunikation zwischen den Visualisierungen mit Hilfe von Animationen, Text und Bildern. Nun weiß Anna auch, wie die verschiedenen Fenster zusammenhängen und kann sich ihrer eigentlichen Aufgabe widmen.

% Was heißt GDP? Wie wird das berechnet? Was soll die Spitze bei 1990? Woher kommt die? Diese eine Komponente scheint kaputt zu sein, wo kann ich mich beschweren?

In der Tabelle findet sie auch eine Spalte \enquote{SNP}. Anna weiß zwar, dass sie die Abkürzung schon einmal gesehen hat, kennt aber im Moment ihre Bedeutung nicht. Praktischerweise ist der Spaltenkopf mit der Wikipedia verlinkt und sie wird sofort auf die entsprechende Seite weitergeleitet. Anna erinnert sich, dass \enquote{SNP} \enquote{Single-nucleotide polymorphism} bedeutet und sie bekommt auch gleich zusätzliche Informationen zu diesem Thema. Sie widmet sich weiter der Tabelle und stellt fest, dass die Ortsbezeichnung \enquote{Kopenhagen} nicht mit der Markierung in der Karte übereinstimmt. Außerdem ist sie erstaunt, wie hoch die Verbreitung eines bestimmten SNPs dort ist und würde gerne die Ursache dafür wissen. In der Hilfe zur Tabelle wurde sie auch über die Möglichkeit, Kommentare an den Daten vorzunehmen, aufgeklärt. Anna kommentiert sowohl die falschen Geokoordinaten als auch ihre Frage über die Verbreitung des SNPs, sodass sie später per Email über Antworten informiert wird. Nun kann Anna sich mit der vierten Visualisierung, dem Balkendiagramm, beschäftigen. Allerdings reagiert es auf keine Mausklicks und macht auch sonst nicht den Eindruck, die Daten akkurat darzustellen. Anna meldet die kaputte Visualisierung über die eingebaute Reporting-Funktion und schließt das Fenster, um sich den anderen drei Visualisierungskomponenten zuzuwenden.

% ###################################################
\section{Anforderungsanalyse}
\label{section:anforderungsanalyse}

Aus dem Szenario (Kapitel~\ref{section:szenario}) lassen sich nun verschiedene Anforderungen an ein Hilfesystem für komposite Informationsvisualisierungssysteme ableiten.

\subsection{Funktionale Anforderungen}
\label{section:funktionale_anforderungen}

Blabla

\begin{itemize}
	\item\textbf{Überblick}: Das Hilfesystem soll einen kurzen Überblick über das InfoVis-System geben und Darstellungsform sowie Inhalt jeder Komponente kurz erläutern.
	\item\textbf{Bedienung}: Das Hilfesystem soll erklären können, wie eine Komponente bedient wird. Diese Informationen umfassen beispielsweise welche Operationen welche Aktionen (eventuell auf welchen Daten) ausführen.
	\item\textbf{Reporting}: Fehler in Komponenten sollen über ein Reporting-System gemeldet werden können.
	\item\textbf{Verlinkung}: Das Hilfesystem soll nicht-triviale Begriffe mit der Wikipedia verlinken, sodass nicht nur auf die Begriffsbedeutung hingewiesen werden, sondern dem Benutzer auch zusätzliche Informationen zur Verfügung gestellt werden können.
	\item\textbf{Kommunikation}: Das Hilfesystem soll erklären können, wie gegebene Komponenten miteinander kommunizieren.
	\item\textbf{Kommentare}: Der Benutzer soll die Möglichkeit haben Daten zu kommentieren und Bereiche der Visualisierung zu markieren und mit ebenfalls mit einem Kommentar zu versehen, sodass auch auf fehlende Daten hingewiesen werden kann.
\end{itemize}

\subsection{Nichtfunktionale Anforderungen}
\label{section:nichtfunktionale_anforderungen}

Bla bla

\begin{itemize}
	\item\textbf{Korrektheit}: Eine gegebene Hilfestellung darf keine Fehlinformationen enthalten, weil sie sonst mehr verwirrt als hilft.
	\item\textbf{Vollständigkeit}: Eine gegebene Hilfestellung muss alle Informationen enthalten, die der Nutzer benötigt um danach seine gewünschte Aufgabe ausführen zu können.
	\item\textbf{Verständlichkeit}: Hilfestellungen müssen in einer Form präsentiert werden, die der Benutzer schnell und mit geringem mentalen Aufwand verarbeiten kann.
	\item\textbf{Einheitlichkeit}: Das Look \& Feel von Teilen des Hilfesystems (z.B. Kommentare) muss komponentenübergreifend einheitlich sein, damit der Benutzer einmal gelerntes wiederverwenden kann.
	\item\textbf{Minimalität}: Der Komponentenentwickler soll seine Komponente mit möglichst wenig Aufwand zum Hilfesystem kompatibel machen können, ansonsten werden nur sehr wenige Komponenten -- und damit der Benutzer -- davon profitieren.
	\item\textbf{Universalität}: Das Hilfesystem soll für alle Komponenten und Visualisierungen in gleicher Qualität funktionieren.
	\item\textbf{Wiederverwendbarkeit}: Die Kommentare sollen möglichst in allen Visualisierungen wiederverwendet werden, damit viele Benutzer von den Erkenntnissen anderer profitieren können.
	\item\textbf{Unaufdringlichkeit}: Das Hilfesystem soll den Benutzer nicht von seinen Aufgaben ablenken und nur auf Anfrage zum Einsatz kommen oder es soll selbständig erkennen, wenn der Benutzer Hilfe benötigt.
\end{itemize}

% ###################################################
\section{Grundlagen}
\label{section:grundlagen}

\subsection{Semantische Datensätze}
\label{section:semantische_daten}

% Grundlagen von RDF, RDFS, OWL. Kann ich mehr oder weniger aus dem Beleg abschreiben.
Eine Ontologie ist \enquote{an explicit specification of a conceptualization} \cite{Gruber1995} und wird benutzt um domänenspezifisches Wissen abzubilden \cite{Chandrasekaran1999}. Sie besteht aus mehreren Elementen:

\begin{itemize}
	\item Eine Klasse repräsentiert ein Konzept, eine Entität, ein Ding, beispielsweise ein \textit{Smartphone}.
	\item Eine Instanz ist ein konkretes Objekt einer Klasse, zum Beispiel das \textit{iPhone mit der Seriennummer XYZ-ABC}.
	\item Datenattribute beschreiben eine Instanz näher, zum Beispiel die \textit{Seriennummer} oder \textit{Bildschirmgröße} des iPhones.
	\item Objektattribute beschreiben Beziehungen zwischen Klassen und deren Instanzen, beispielsweise eine Person \textit{besitzt} ein Smartphone.
	\item Außerdem existieren noch Axiome, Regeln, Funktionen und Einschränkungen, welche die Logik einer Ontologie beschreiben.
\end{itemize}

Um eine Ontologie maschinenlesbar darzustellen, hat das World Wide Web Consortium (W3C) verschiedene Beschreibungssprachen eingeführt. Die bekanntesten sind Resource Description Framework Schema (RDFS) und Web Ontology Language (OWL). Mit diesen Sprachen lässt sich unterschiedlich viel Semantik u.\,a. in Form von Ontologien, Thesauren oder Vokabularen ausdrücken; die Komplexität der Dokumente und damit der Aufwand, sie zu erstellen, verhalten sich aber direkt proportional (siehe Abbildung~\ref{figure:grundlagen-semantic_spectrum}) \cite{Bergman2007}.

\begin{figure}[htbp]
	\centering
	\includegraphics[width=0.75\textwidth]{images/grundlagen-semantic_spectrum.png}
	\caption{Semantic Spectrum}
	\label{figure:grundlagen-semantic_spectrum}
\end{figure}

%TODO Noch auf OWL oder RDF/RDFS eingehen? Für diese Arbeit ist es ja eigentlich egal.


\subsection{Informationsvisualisierung}
\label{section:informationsvisualisierung}

% Grundlagen von InfoVis und Beispiele
Card et al. \cite{Card1999} definieren den Begriff \enquote{Informationsvisualisierung} wie folgt:

\begin{quote}
The use of computer-supported, interactive, visual representations of abstract data to amplify cognition.
\end{quote}

Beispiele dafür sind Balkendiagramme, Treemaps \cite{Shneiderman1992} und Parallel Coordinates \cite{Inselberg1991} (Abbildung~\ref{figure:parallel_coordinates}). Mangels Interaktivität sind Infografiken \cite{Smiciklas2012} von den eben definierten Informationsvisualisierungen ausgeschlossen.

\begin{figure}[htbp]
   \centering
   \includegraphics[width=0.5\textwidth]{images/grundlagen-parallel_coordinates.png} 
   \caption{Parallel Coordinates}
   \label{figure:parallel_coordinates}
\end{figure}

Im Folgenden wird ein Überblick über verschiedene Informationsvisualisierungen gegeben. Dieser orientiert sich an Keim \cite{Keim2002}, welcher Informationsvisualisierungen nach dargestellten Daten, Visualisierungs- und Interaktionstechnik klassifizierte. Die Abbildungen stammen, sofern nicht anders angegeben, aus \cite{Heer2010}.

\subsubsection{Dargestellte Daten}
\label{section:dargestellte_daten}

\textbf{Eindimensionale Daten} sind beispielsweise Zeitreihen, also Folgen von Daten (1992, 1993, 1995...). Nach Keim können diese aber mit anderen Datenobjekten assoziiert sein. Beispiele für InfoVis eindimensionaler Daten wären demnach ein Index Chart (Abbildung~\ref{figure:index_chart}) oder eine einfache Timeline.

\begin{figure}[htbp]
   \centering
   \includegraphics[width=0.5\textwidth]{images/grundlagen-index_chart.png}
   \caption{Index Chart}
   \label{figure:index_chart}
\end{figure}

\textbf{Zweidimensionale Daten} haben zwei unterschiedliche Dimensionen, wie zum Beispiel eine Geokoordinate (geografische Länge und Breite). Beispiele für InfoVis dieser Daten sind eben Karten (Abbildung~\ref{figure:karte}) oder häufig verwendete zweidimensionale Visualisierungen wie z.\,B. ein Balkendiagramm.

\begin{figure}[htbp]
   \centering
   \includegraphics[width=0.5\textwidth]{images/grundlagen-karte.png}
   \caption{Karte}
   \label{figure:karte}
\end{figure}

\textbf{Multidimensionale Daten} haben demnach mehr als zwei unterschiedliche Dimensionen, typischerweise komplexe Objekte wie Autos (Hubraum, Maximalgeschwindigkeit, Leistung, Benzinverbrauch...) oder Digitalkameras (Megapixel, Sensorgröße, maximale Lichtempfindlichkeit, Gewicht...). Um diese Daten darzustellen, werden oft Parallel Coordinates (Abbildung~\ref{figure:parallel_coordinates}) oder Scatterplot Matrizen (Abbildung~\ref{figure:scatterplot_matrix}) verwendet.

\begin{figure}[htbp]
   \centering
   \includegraphics[width=0.5\textwidth]{images/grundlagen-scatterplot_matrix.png}
   \caption{Scatterplot Matrix}
   \label{figure:scatterplot_matrix}
\end{figure}

\textbf{Text} kann erst nach einer Vorverarbeitungsphase mit Zahlen beschrieben werden (z.\,B. Wörter zählen), ansonsten schlagen herkömmliche Visualisierungsansätze fehl. Ein im Web verbreitetes Beispiel ist die Tag Cloud (Abbildung~\ref{figure:tag_cloud}\footnote{\url{http://4.bp.blogspot.com/-WvicpJ9QqQs/TpbqvbKhX3I/AAAAAAAADGc/3PczLY2P0xs/s1600/uni_tag_cloud_wordle.png}}).

\begin{figure}[htbp]
   \centering
   \includegraphics[width=0.5\textwidth]{images/grundlagen-tag_cloud.png}
   \caption{Tag Cloud}
   \label{figure:tag_cloud}
\end{figure}

\textbf{Hierarchien und Netzwerke} beschreiben Relationen und Verbindungen zwischen Objekten. Ein Beispiel für InfoVis von Hierarchien ist der klassische Baum (Abbildung~\ref{figure:baum}), für Netzwerke ein Node-Link-Diagramm (Abbildung~\ref{figure:node-link-diagramm}).

\begin{figure}[htbp]
   \centering
   \includegraphics[width=0.75\textwidth]{images/grundlagen-baum.png}
   \caption{Baum}
   \label{figure:baum}
\end{figure}

\begin{figure}[htbp]
   \centering
   \includegraphics[width=0.25\textwidth]{images/grundlagen-node-link-diagramm.png}
   \caption{Node-Link-Diagramm}
   \label{figure:node-link-diagramm}
\end{figure}

% Am Schluss hat Keim noch Software/Algorithmen, aber ich sehe den Unterschied zu Multidimensionalen Daten nicht. Außerdem kann man - wenn schon mal grundlos neue Kategorien eingeführt werden - dann auch Produktionsprozesse oder what not gesondert betrachten

\subsubsection{Visualisierungstechniken}
\label{section:Visualisierungstechniken}

\textbf{Standard 2D/3D} Visualisierungen beinhalten Balkendiagramme, Liniendiagramme, sowie andere zweidimensionale Plots und Karten. Ein Beispiel dafür ist das Index Chart (Abbildung~\ref{figure:index_chart}).

\textbf{Multidimensionale} Visualisierungen\footnote{Die Bezeichnung stammt von Chi \cite{Chi2000} und wird verwendet, da sie logischer erscheint als Keims \enquote{geometrically transformed displays}.} sind Darstellungen multidimensionaler Datensätze jeder Art. Beispiele sind Parallel Coordinates (Abbildung~\ref{figure:parallel_coordinates}) und die Scatterplot Matrix (Abbildung~\ref{figure:scatterplot_matrix}).

\textbf{Symbolische} Visualisierungen setzen auf verschiedene Art und Weise Symbole ein. Das können auf eine Karte projizierte Kuchendiagramme sein (Abbildung~\ref{figure:symbol_map}) oder Smileys, die abhängig von den Daten lächeln oder weinen \cite{Chernoff1973}.

\begin{figure}[htbp]
   \centering
   \includegraphics[width=0.5\textwidth]{images/grundlagen-symbol_map.png}
   \caption{Symbol Map}
   \label{figure:symbol_map}
\end{figure}

\textbf{Dense Pixel} Visualisierungen assoziieren jeden Wert einer Dimension mit einem eingefärbten Pixel und platzieren die Pixel einer Dimension nebeneinander. Ein Beispiel dafür ist das Recursive Pattern \cite{Keim1995} (Abbildung~\ref{figure:recursive_pattern}).

\begin{figure}[htbp]
   \centering
   \includegraphics[width=0.5\textwidth]{images/grundlagen-recursive_pattern.png}
   \caption{Recursive Pattern}
   \label{figure:recursive_pattern}
\end{figure}

\textbf{Verschachtelte} Visualisierungen repräsentieren Hierarchien, wobei Kindknoten innerhalb ihrer Eltern dargestellt werden. Beispiele dafür sind Treemaps \cite{Shneiderman1992} oder Nested Circles (Abbildung~\ref{figure:nested_circles}).

\begin{figure}[htbp]
   \centering
   \includegraphics[width=0.25\textwidth]{images/grundlagen-nested_circles.png}
   \caption{Nested Circles}
   \label{figure:nested_circles}
\end{figure}

\subsubsection{Interaktionstechniken}
\label{section:interaktionstechniken}

\textbf{Dynamische Projektionen} zeigen dem Benutzer automatisch beispielsweise verschiedene Scatterplots des Datensatzes. Diese Interaktionstechnik eignet sich besonders für multidimensionale Datensätze. Umgesetzt wurde sie zum Beispiel in XGobi \cite{Swayne1998}.

Durch \textbf{interaktives Filtern} bestimmt der Benutzer, welche Teilmenge des Datensatzes visualisiert wird. Das kann durch direktes Auswählen (Browsing) oder durch Bestimmen von Eigenschaften der gewünschten Daten (Querying) passieren. Letzteres ist in modernen E-Commerce-Systemen durch Facetten \cite{Yee2003} umgesetzt (Abbildung~\ref{figure:faceted_browsing}).

\begin{figure}[htbp]
   \centering
   \includegraphics[width=0.5\textwidth]{images/grundlagen-faceted_browsing.png}
   \caption{Faceted Browsing}
   \label{figure:faceted_browsing}
\end{figure}

\textbf{Interaktives Zoomen} ermöglicht es dem Benutzer mehr oder weniger Details anzuzeigen. Damit ist nicht nur der computergrafische Vorgang der Skalierung gemeint (wie etwa bei einem Mikroskop), sondern auch schlecht sichtbare Elemente auszublenden (wie z.\,B. bei Google Maps, siehe Abbildung~\ref{figure:zoom}).

\begin{figure}[htbp]
   \centering
   \includegraphics[width=0.5\textwidth]{images/grundlagen-zoom.png}
   \caption{Interaktiver Zoom bei Google Maps: Im rechten, ausgezoomten Bild fehlt beispielsweise die Interstate 278 (grüne Markierung)}
   \label{figure:zoom}
\end{figure}

% distortion (fisheyes and such)

% link & brush

\subsection{User Assistance}
\label{section:user_assistance}

% wie funktioniert überhaupt sensemaking, also wissensaufnahme?

% Hilfe für Devs wird hier ausgenommen, weil das System ja auf Endnutzer ausgelegt sein wird

% End-User Hilfe in Desktop Software: Karl Klammer, F1, ?-Button, FAQ, First Steps, Tutorials, Manual

% End-User Hilfe im Web: Intros, sonst eher gar nix. User wird geholfen durch ausprobieren und Semantic Transparency. Und Adaptivität, Personalisierung.

% intelligente, automatische, what not user assistance

% ###################################################
\chapter{Konzeption}
\label{chapter:konzeption}


%#####################################################
\printbibliography[title=Literaturverzeichnis]

%\printindex

\end{document}